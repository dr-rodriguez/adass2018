% super simple template for automated 2018 ADASS manuscript generation from the registration entry
% most comments have been removed, see the ADASS_template.tex for a fully commented version

% Version 4-nov-2018 (Peter Teuben)

\documentclass[11pt,twoside]{article}
\usepackage{asp2014}

\aspSuppressVolSlug
\resetcounters

\bibliographystyle{asp2014}


\markboth{Dowler et al.}{Archive-2.0: Metadata and Data Synchronisation} 

\begin{document}

\title{Archive-2.0: Metadata and Data Synchronisation between MAST, CADC, and ESAC}


\author{Patrick~Dowler,$^1$, Maria~Arevalo,$^2$, Adrian~Damian,$^1$ Javier~Duran,$^2$ Daniel~Durand,$^1$
Severin~Gaudet,$^1$ Jonathan~Hargis,$^3$ Brian~Major,$^1$ Brian~McLean,$^3$ Oliver~Oberdorf,$^3$ and David~R.~Rodriguez$^3$}

\affil{$^1$National Research Council Canada, Victoria, British Columbia, Canada}
\affil{$^2$European Space Astronomy Centre, ???, ???, Spain}
\affil{$^3$Space Telescope Science Institute, Baltimore, Maryland, United States of America}


\paperauthor{Patrick~Dowler}{pdowler.cadc@gmail.com}{https://orcid.org/0000-0001-7011-4589}{National Research Council Canada}{Canadian Astronomy Data Centre}{Victoria}{British Columbia}{V9E 2E7}{Canada}
\paperauthor{Maria~Arevalo}{Author2Email@email.edu}{ORCID_Or_Blank}{Author2 Institution}{Author2 Department}{City}{State/Province}{Postal Code}{Country}
\paperauthor{Adrian~Damian}{Author2Email@email.edu}{ORCID_Or_Blank}{Author2 Institution}{Author2 Department}{City}{State/Province}{Postal Code}{Country}
\paperauthor{Javier~Duran}{Author2Email@email.edu}{ORCID_Or_Blank}{Author2 Institution}{Author2 Department}{City}{State/Province}{Postal Code}{Country}
\paperauthor{Daniel~Durand}{Author2Email@email.edu}{ORCID_Or_Blank}{Author2 Institution}{Author2 Department}{City}{State/Province}{Postal Code}{Country}
\paperauthor{Severin~Gaudet}{Author2Email@email.edu}{ORCID_Or_Blank}{Author2 Institution}{Author2 Department}{City}{State/Province}{Postal Code}{Country}
\paperauthor{Jonathan~Hargis}{jhargis@stsci.edu}{https://orcid.org/0000-0002-8722-9806}{Space Telescope Science Institute}{Mikulski Archive for Space Telescopes}{Baltimore}{MD}{21218}{USA}
\paperauthor{Brian~Major}{Author2Email@email.edu}{ORCID_Or_Blank}{Author2 Institution}{Author2 Department}{City}{State/Province}{Postal Code}{Country}
\paperauthor{Brian~McLean}{mclean@stsci.edu}{https://orcid.org/0000-0002-8058-643X}{Space Telescope Science Institute}{Mikulski Archive for Space Telescopes}{Baltimore}{MD}{21218}{USA}
\paperauthor{Oliver~Oberdorf}{ooberdorf@stsci.edu}{}{Space Telescope Science Institute}{Mikulski Archive for Space Telescopes}{Baltimore}{MD}{21218}{USA}
\paperauthor{David~R.~Rodriguez}{drodriguez@stsci.edu}{https://orcid.org/0000-0003-1286-5231}{Space Telescope Science Institute}{Mikulski Archive for Space Telescopes}{Baltimore}{MD}{21218}{USA}

% leave these commented for the editors to enable them
%\aindex{Dowler,~P.}
%\aindex{Coauthor,~A.}          % remove and add as you need
  
\begin{abstract}

The Canadian Astronomy Data Centre (CADC) and the European Space Astronomy Centre (ESAC) maintain mirrors of and provide user access to the HST data collection. A new mirroring approach was needed to improve consistency and support future missions like JWST. The Common Archive Observation Model (CAOM) is used as the core model for all data holdings at the CADC and the Mikulski Archive for Space Telescopes (MAST) and was extended to support a metadata and data synchronization system.

The metadata synchronization process relies on a simple RESTful web service operated by the metadata source (MAST) and a metadata harvesting tool run by the mirror centres (CADC and ESAC). The data synchronization process uses  CAOM metadata to discover and retrieve files from the source (MAST) to the mirror sites.  Through the use a backend plugins, the CADC and ESAC have extended the file synchronization tool to interface with their respective site-specific storage systems.

Using the common metadata model, services and tools as a base, partners can augment their own system with additional information specifically intended to provide added value features. ESAC provides information about publications in their instance of the Archive and both CADC and ESAC provide additonal IVOA services for users.
  
\end{abstract}

\section{Introduction}

TODO: outline following sections

\section{CAOM Enhancements}

This project required several enhancements to CAOM and resulted in the release of CAOM version 2.3 in 2018.

Several changes were designed to improve extensibility so that metadata curators could better describe their content and use CAOM as their primary data model.

We lifted a constraint on the algorithm name for a SimpleObservation so that other values (e.g. simulation) could be used and retained expsoure as the default value.

CalibrationLevel was modified to support new values in IVOA ObsCore \citep{std:ObsCore} (calib\_level) and extended to include planned observations. The DataProductType was converted from an enuemration to an extensible vocabulary with the base vocabulary terms taken from ObsCore (dataproduct\_type); as a vocabulary, new terms can be introduced simply by providing a resolvable URI (typically an http URL) refering to the definition of the new term(s). The ProductType enumeration was also converted to a vocabulary with the base terms from the CAOM-2.2 enumeration values.

Polygon validity (used to describe the spatial coverage of an observation) was restricted to be consistent with IVOA DALI \citep{std:DALI} polygon (simple polygon with counterclockwise winding direction). The previous polymorphic shape and general polygon with disjoint parts was retained to provide a more detailed footprint when necessary.

A content checksum was added to the Artifact class to support data Synchronisation. In CAOM checksums are expressed as URIs where the scheme is the checksum algorith and the scheme-specific part is an ascii representation (hexidecimal) of the value. We are currently using MD5 checksums for file verification as it provides a good balance of robustness and performance and was supported already by the CADC archive storage system.

In order to support robust metadata synchronisation and validation, we defined a metadata checksum algorithm so that each serialised entity included a stable checksum of all metadata fields and an accumulated metadata checksum of itself and all child entities. The accumulated metadata checksum of the Observation thus captures the state of the entire structure and can be used to verify serialisation, transmission, and deserialisation of a complete observation. This checksum protects against a variety of software bugs: lossy storage of numeric values in databases and files, process flaws like updating database values without updating corresponding checksums, inconsistent reading and writing of XML documents, inconsistent handling of lists or ordering of sets, and was even used to detect and fix a race condition effecting database updates.

\section{Metadata Synchronisation}

Metadata synchronisation involves an application (caom2harvester) and a web service that implements a simple REST \citep{REST} API. The web service API supports an endpoint with path elements for the collection and the observationID from CAOM (/observations/{collection}/{observationID}). A GET request to the collection returns a list of observations; optional parameters allow control of start and end timestamp values and to limit the number of records in the list. The listing includes the collection, observationID, modification timestamp, and accumulated metadata checksum of the observation  in order of increasing modification timestamp. A GET request to the complete path returns a CAOM  observation document (XML). There is a similar endpoint that provides a list of deleted observations (/deleted/{collection}); the output format is similar to the observation listing, but it includes the internal UUID of the observation and does not include a metadata checksum.

The caom2harvester tool harvests a single collection at a time and normally operates in incremental mode (recent changes) to maintain an up-to-date copy of the metadata. The read-only web service is implemented at MAST while CADC and ESAC operate the caom2harvester application to pull metadata updates from MAST and store in a local database (currently PostgteSQL). For the project, CADC and ESAC could harvest metadata from MAST at approximately 50K observations per hour (significantly faster than the MAST pipeline could generate them) so keeping up to date was easily accomplished with a single cron job.

The caom2harvester keeps track of current harvest state (last timestamp seen) and all failures in the destination database, so it can be killed or crash and be restarted with no difficulty. A retry mode can be used to reharvesting of tracked failures. The outcome of a retry will be one of fix or delete or fail again.

The caom2harvester tool also supports a validation mode to check and maintain the consistency of the entire metadata collection. This mode gets a complete listing from the source (REST API) and local (database) and makes three comparisons:
\begin{itemize}
\item{missing observations (in source - not in destination)}
\item{missed deletions (not in source - in destination)}
\item{accMetaChecksum mismatch (cause: persistence or serialisation bug in source or destination or harvest is not up to date)}
\end{itemize}

In practice, incidents detected by validation are used to create new failure records and then the retry mode is used to reharvest those and bring the destination database up to the correct state.

\section{Data Synchronisation}

For this project, we implemented a new file synchronisation tool named caom2-artifact-sync that used the local CAOM database to discover files to retrieve from the source. The discovery mode of this tool uses the same observation timestamps to scan new or changed observations and figure out which artifacts (files) to retrieve. To do this, the tool makes use of a local storage plugin (plugin API defined) to check thee current file status (existence and checksum).

Like the metadata harvesting tool, file synchronisation normally operates in incremental mode: it uses local CAOM metadata to discover new or modified files and schedule downloads. A separate mode performs downloads. A validation mode performs a full comparison of files referenced in CAOM with those in the local storage system and (optionally) schedules downloads to fix any discrepancies. Since CADC and ESAC are only able to servce public data, downloads are initially scheduled for the data release date found in the CAOM metadata. Failed downloads are re-scheduled with a delay that depends on the type of error encountered). The download mode is multi-threaded and we found that 16-48 threads were needed to attain good network utilisation, largely due to overheads in retrieving small files.

After some network tuning at MAST, CADC and ESAC were able to retrieve ~30TiB per day for the initial downloads; incremental updates due to reprocessing at MAST are picked up and synchronised with minimal effort.

The caom2-artifact-sync tool also supports validation mode. This mode compares the local CAOM artifact list to storage content to detect missing files or files with checksum mismatch (reschedule download) and to detect orphaned files in storage (due to a rename at MAST or artifact being removed from CAOM) and schedule deletion.

\section{Data Delivery to Users}

CADC and ESAC provide HTTP download service that uses local CAOM database to determine: if the file exists, if the file is readable by the user (public in the case os HST), and if the local copy is up-to-date (Artifact.contentChecksum == local storage checksum): deliver the file. Otherwise, if the file exists we redirect the download request to the MAST download service (same on used in data sync above)

Future work: All partner sites could use client location (geo-ip) to redirect downloads to another partner, thus implementing a sort of content distribution network (CDN) for shared and synchronised data collections.

\section{Site-Specific Enhancements}

For operational support, CADC makes the table of metadata harvest failures visible via a TAP \citep{std:TAP} service; MAST can diagnose and fix metadata issues and once incremental harvesting pciks up the changes the failures are automatically removed. CADC also makes the table of scheduled downloads (including data sync failures) visible via the same TAP service; MAST can diagnose and fix data delivery issues and eventually see the results.

For users, MAST makes publication metadata available and ESAC harvests this and combines it with the CAOM metadata in the EHST interface.

The MAST portal uses the CAOM database to search the HST metadata. The CADC AdvancedSearch and the ESAC EHST portals provide alternate browser-based user interfaces to search the HST metadata.

CADC provides IVOA services (TAP, SIA \citep{std:SIA}, DataLink \citep{std:DataLink}, SODA \citep{std:SODA}) and IVOA ObsCore data model support as part of their programmatic archive data discovery services. ESAC also provides IVOA services (TAP, SIA, SSA \citep{std:SSA}) based on the HST metadata in their CAOM database.

\bibliography{P10-5}

\end{document}

